\documentclass[12pt]{article}
\usepackage[spanish]{babel}
\usepackage[utf8]{inputenc}
\usepackage{float}
\usepackage{hyperref}
\usepackage{graphicx}
\usepackage{listings}%Para código bonito
\usepackage{apacite}
\deactivatequoting
\begin{document}
\hypersetup{pageanchor=false}
\begin{titlepage}
\begin{center}
\vspace{2cm}
\begin{figure}[H]
\begin{center}
\includegraphics[width=7cm]{logo_fes.png}\\
\includegraphics[width=7cm]{Flisol.png}
\end{center}
\end{figure}
\Huge{\textbf{TEMARIO:}}\\
\Huge{\textbf{Taller básico de \LaTeX}}
%\vspace{0.6cm}
\rule{100mm}{0.1mm}\\
\begin{Large}
Ponente: Víctor Hugo Vázquez Montoya
\href{mailto:victor.hugovz@comunidad.unam.mx}{victor.hugovz@comunidad.unam.mx} 
\end{Large}
\vspace{2cm}
\end{center}
\end{titlepage}
\hypersetup{pageanchor=true}
\section*{Objetivo:}
Los participantes adquirirán la habilidad de hacer documentos científicos de alta calidad tipográfica con el fin de producirlos 
por sí mismos.
\section*{Temas}
\begin{enumerate}
\item \textbf{\large {Primeros pasos en \LaTeX}}
\vspace{5mm}

1.1. ¿Qué es \LaTeX\, y porqué se considera software libre?\\
1.2. Instalación de un entorno integrado de edición para \LaTeX \,: Kile Comandos en la terminal para su instalación\\
1.3. Conociendo Kile: Menús y barras de herramientas\\
1.4. ¿Qué hago si no pude instalar Kile?. \LaTeX\, en la nube: \href{https://www.overleaf.com/}{Overleaf}\\
\item \textbf{\large {La estructura de un documento en \LaTeX}}
\vspace{5mm}

2.1. ¿Cómo escribir un documento desde el inicio?: El comando 
\begin{lstlisting}
\documentclass[options]{style}
\end{lstlisting}
2.2. Paquetería escencial: El preámbulo\\
2.3. Errores comunes al compilar mi documento\\
2.4. Tipos de documentos en \LaTeX \\
2.5. Distintos tipos y tamaños de fuentes.\\
2.6 Listado y enumeración de elementos con viñetas\\
2.7. Secciones, subsecciones, capítulos: Elaboración automática de índices y de una plantilla
\item \textbf{\large {Entornos matemáticos}}
\vspace{5mm}

3.1. Paquetería básica en el preámbulo para escribir ecuaciones\\
3.2. Ecuaciones referenciadas con etiquetas en el texto previo y posterior\\
3.3. Entornos matemáticos simples 
\pagebreak
\item \textbf{\large {Tablas y figuras}}
\vspace{5mm}

4.1. Paquetería escencial para hacer tablas\\
4.2. Posicionamento de tablas\\
4.3. Un propio índice de tablas\\
4.4. Paquetería escencial para insertar figuras\\
4.5. Gráficos con texto y sin texto\\
4.6  Un propio índice de figuras
\item \textbf{\large {Bibliografías artesanales}}
\vspace{5mm}

5.1. Paquetería en el preámbulo para la bibliografía\\
5.2. Notas al pie de página\\
5.3. Los archivos \textbf{.bib} y su estructura\\
5.3. Formas de citación: Estilos y los administradores de referencias\\
5.4. Bibliografía en formato APA: Paquetería y comandos
\end{enumerate}
\textbf{Sugerencias de evaluación para el aprendizaje}
\begin{itemize}
 \item Desarrollo de ejemlos específicos de cada uno de los subtemas
 \item Utilizar recursos didácticos en línea
 \item Revisión constante por parte del ponente a las dudas que surjan durante el taller

\end{itemize}
\bibliographystyle{apacite}
\nocite{*}
\bibliography{references}
\end{document}
%fin